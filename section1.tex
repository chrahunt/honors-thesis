%!TEX root=thesis.tex
\section{Local Surface Theory}

Investigating the properties of surfaces will form the basis on which our journey towards minimal surfaces rests. We will go through the perils inherent in defining surfaces, and then how we use calculus on these surfaces to approximate the behavior at a point and extract information about the surface by extending on that approximation. Along the way we will make and derive various tools used for analyzing these surfaces, some of which we will bring along with us to Minimal Surfaces, others we will not, but they are interesting nonetheless and help illustrate some of the connections between Differential Geometry and various other branches of mathematics. We will then look at parallels between these surfaces in $\RR^3$ and higher dimensional manifolds, though only briefly.

\subsection{Defining Surfaces in $\RR^3$}

Defining surfaces is not particularly difficult, but our methods require a different set of considerations. We will first define a simple surface, or coordinate patch, which represents a small open section of a larger surface. After considering that, we build surfaces using an open covering of these coordinate patches, with a special requirements on their intersections which, we will see, is necessary to ensure our tools later on function properly.

