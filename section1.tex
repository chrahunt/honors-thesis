%!TEX root=thesis.tex
\section{Surface Theory}

Investigating the properties of surfaces will form the basis on which our journey towards minimal surfaces rests. We will brave the perils that come in defining surfaces, and then use calculus on these surfaces to approximate the behavior at a point and extract information about the surface by extending on that approximation. Along the way we will make and derive various tools used for analyzing these surfaces, some of which we will bring along with us to Minimal Surfaces, others we will not, but they are interesting nonetheless and help illustrate some of the connections between Differential Geometry and various other branches of mathematics. We will look at parallels between these surfaces in $\RR^3$ and higher dimensional manifolds, though only briefly.

\subsection{Defining Surfaces in $\RR^3$}

% Simple surface
% TODO: Check to see if U needs to be connected.
% TODO: Fix emphasis here, the wrong parts are italicized.
Defining surfaces is not particularly difficult, but our methods require a different set of considerations. We will first define a simple surface, or coordinate patch, which represents a small open section of a larger surface. After considering that, we define a geometry-preserving we build surfaces using an open covering of these coordinate patches, with a special requirements on their intersections which, we will see, is necessary to ensure that the conclusions we draw are independent of the specific covering chosen for the surface.

\begin{defn}
  Let $\sU$ be an open subset of $\RR^2$, then a function $\bx: \sU \to \RR^3$, written $\bx(u_1, u_2)$, is known as a \emph{simple surface} if it is injective and regular. The condition for regularity is that
  \[
    \frac{\partial\bx}{\partial u_1} \times \frac{\partial\bx}{\partial u_2} \neq 0
  \]
  across the domain of $\bx$. The function $\bx$ may also be referred to as a \emph{coordinate patch}.
\end{defn}

This definition is exactly what we want. Conceptually, the regularity of $\bx$ ensures that the mapping of $\sU$ to $\RR^3$ doesn't create any sharp creases or corners, but perhaps more relevant to our purposes, it also ensures that the partial derivatives form a linearly independent set. This will be critical later on.

% Monge Patch example
% TODO: Refactor and incorporate smooth/one-to-one
Before going any further we introduce the \emph{monge patch}, a simple surface that will be instrumental as a simple example on which we can carry out our calculations. Let $f(x, y) = z$ be a function, then we can use this in the parameterization of our patch as follows: $\bx(u, v) = (u, v, f(u, v))$. Notice that
\begin{align*}
  \frac{\partial\bx}{\partial u} \times \frac{\partial\bx}{\partial v} &= \left(-\frac{\partial f}{\partial u}, -\frac{\partial f}{\partial v}, 1\right)\\
  &\neq 0
\end{align*}
so it really is a simple surface.

% TODO: Fix akward wording
Another example we can introduce is a \emph{surface of revolution}. Given a curve in the $(r, z)$ with $r = r(t) > 0$ and $z = z(t)$, the surface of revolution generated by this curve (via rotation about the z-axis, is given by
\[
  \bx(t, \theta) = (r(t)\cos\theta, r(t)\sin\theta, z(t)) \ .
\]

% NOTE: p15
Assuming the curve is regular and injective, we can show that a surface of revolution really is a simply surface by simply checking. (A curve, say $\ba : (a, b) \to \RR^3$ is considered \emph{regular} when $\frac{d\ba}{dt} \neq 0$ for any $t$ in $(a, b)$.)

A quick calculate yields the following:
\begin{align*}
  \bx_1 &= (\dot{r}\cos{\theta}, \dot{r}\sin{\theta}, \dot{z})\\
  \bx_2 &= (-r(t)\sin{\theta}, r(t)\cos{\theta}, 0)\\
  \bx_1 \times \bx_2 &= (-\dot{z} r(t) \cos{\theta}, -\dot{z} r(t) \sin{\theta}, \dot{r}r(t))
\end{align*}
which is not $0$ since $r(t) > 0$, and then the original curve was regular so $\dot{r}, \dot{z} \neq 0$

% NOTE: Coordinate Transformation Definition (p79)
\begin{defn}
  % TODO: Refactor
  Let $\sU, \sV$ be open subsets of $\RR^2$. A $C^k$ \emph{coordinate transformation} is an injective $C^k$ function $f: \sV \to \sU$ whose inverse $g:\sU \to \sV$ is also of class $C^k$. We call a coordinate transformation \emph{continuous} if ... and \emph{proper} if ...
\end{defn}

In the propositions that follow we will see that the coordinate transformation preserves some of the geometric properties 

% NOTE: Surface definition (p89)
\begin{defn}
  A $C^k$ \emph{surface} in $\RR^3$ is a subset $M \subset \RR^3$ such that for every point $P \in M$ there is a proper $C^k$ coordinate patch whose image is in $M$ and which contains an $\eps$-neighborhood of $P$ for some $\eps > 0$. Furthermore, if both $\bx: \sU \to \RR^3$ and $\by: \sV \to \RR^3$ are such coordinate patches with $\sU^\prime = \bx(\sU), \sV^\prime = \by(\sV)$, then $y^{-1} \circ \bx: (x^{-1}(\sU^\prime \cap \sV^\prime)) \to (y^{-1}(\sU^\prime \cap \sV^\prime))$ is a $C^k$ coordinate transformation.
\end{defn}

\subsection{Linear Algebra on Surfaces}

% TODO: Explanation of title and the concepts that we will be introducing
\begin{defn}
  We consider two specific tangent vectors on a surface, the 
\end{defn}

\begin{unno_rem}
  We typically consider the partial derivative as the tangent vector of a curve holding one parameter of the function constant, so planting the base of the vector at the point at which it is evaluated.
\end{unno_rem}

% Continuity

% Properness

\begin{defn}
  The \emph{tangent plane} to a simple surface  $\bx: \sU \to \RR^3$ at the point $P = \bx(a, b)$ is the plane through $P$ perpendicular to $\bx_1(a, b) \times \bx_2(a, b)$. The \emph{unit normal} to the surface at $P$ is $\bn(a, b) = \frac{\bx_1 \times \bx_2}{\left|\bx_1 \times \bx_2\right|}$, where the right-hand side is evaluated at $(a, b)$. Note that $\bn(a, b)$ exists because $\bx_1 \times \bx_2 \neq 0$. It is perpendicular to the tangent plane at $P$.
\end{defn}

\subsection{Tensors and Operators on Surfaces}

\subsection{Curvature}

\subsection{Manifolds}