%!TEX root=thesis.tex
\section{Weierstrass Representations for Minimal Surfaces}

  As we saw in the last section, there was more than one way to build up the concept of a Minimal Surface. This is no coincidence. The concept of minimal surfaces as an area-minimizing structure makes it well-suited to describe a number of physical phenomena, and further, its numerous definitions just serve to demonstrate that minimal surfaces lie at the intersection of a number of branches of mathematics. Our goal in this section is to look at what some would consider a more fundamental picture of minimal surfaces, showing their relationship to complex analysis through harmonic functions and culminating in the Weierstrass-Enneper Representation.

  We will start by showing that for any minimal surface we may find \emph{isothermal parameters}, which guarantees for us specific behavior of the metric tensor. Following this we will break and visit some concepts, definitions, and theorems from complex analysis and take a brief look at harmonic functions. Then, since we can find isothermal parameters for any minimal surface, this will imply a relationship between minimal surfaces and their parameter functions which will give us a relatively simply method of \emph{generating} minimal surfaces. The Weierstrass-Enneper Representation then is just a way to satisfy the requirements for generation of a minimal surface, as we shall see.

  % TODO: Citations below
  Throughout, we will use the notation and language acquired from Oprea, as in Section~\ref{ss:opreaSrf} and take~\cite{Opr07} and~\cite{Opr00} as general reference for this section.

\subsection{Background}
  Switching gears significantly from the previous sections, we will take a moment to fill in some background material.

  % TODO: Consider putting this after the harmonic function section? Maybe.
  Recall that any complex value $z$ can be written in the form $z = u + iv$, where $u$ and $v$ are both real values. Likewise, given a function $f(z)$, we can rewrite this as $f(z) = \phi(z) + i\psi(z)$ and combined with the above we have $f(u, v) = \phi(u, v) + i \psi(u, v)$ where $\phi(u, v)$ and $\psi(u, v)$ are both real-valued functions. Throughout, we will use the notation
  \begin{align*}
    \Re z &= u\\
    \Im z &= v\\
    \Re f(z) &= \phi(z) = \phi(u, v)\\
    \Im f(z) &= \psi(z) = \psi(u, v) \ , 
  \end{align*}
  with $\Re$ denothing the \emph{real} part of the variable or function and $\Im$ the so-called \emph{imaginary} portion.

  % TODO: Holomorphic equation

  % TODO: Harmonic functions
  % TODO: Harmonic function definition
  % TODO: Harmonic conjugates
  
  % TODO: briefly mention isothermal parameters

  % TODO: Corr 4.5.7

  % TODO: Isothermal parameters
  \begin{defn}
    A surface 
  \end{defn}

\subsection{Weierstrass Representation}
  % TODO: Build first method
\subsection{Examples}