%!TEX root=thesis.tex
\section{Minimal Surfaces}

Minimal surfaces are found at the intersection of a number of different areas in mathematics. A side-effect of this is a great number of characterizations for the definition of a minimal surface. We will first define minimal surfaces using the language of surface theory that we developed in the previous section, then we will continue in that same vein to view various properties of minimal surfaces using the tensors and concepts developed previously. Having done that we will take a slightly different approach to build up minimal surfaces, as surfaces of least area, which will shape and prepare us for the discussion of the next section.

\subsection{Minimal Surfaces in the Language of Surface Theory}

Minimal surfaces are surfaces with mean curvature everywhere equal to zero. Without an intuitive idea about mean curvature and the implications this seems a dry, technical statement, but the reality couldn't be further from the truth. Recall the definition of mean curvature from the previous section.

% TODO: Square L
With $\wg$ denoting the Weingarten map, we have the mean curvature $H$ at a point $p \in M$ as
\[
  H = \frac{1}{2} (\kappa_1 + \kappa_2) =  \frac{1}{2}\text{trace}(\wg) \ ,
\]
where $\kappa_i$ are the principle curvatures at $p$.

% Minimal surface equation

% Example of calculating mean curvature for a surface, probably the catenoid.

% Showing that a surface of revolution that is a minimal surface must be a catenoid.


\subsection{Properties of Minimal Surfaces}

% Curvature

% Geodesics

% Any other properties?

\subsection{A Different Approach to Minimal Surfaces}
\label{ss:opreaSrf}
% TODO: Citation
% l, m, n from Oprea07 (p88)
% E, F, G, l, m, n, shape operator, H, K, etc.
In the language of the previous section our examination of minimal surfaces would be something of a slog. The properties and intuition that we will build up will cover a wide range of approaches from the strictly geometric (in which case, for ease of understanding we will use the more intuitive notation of the former) to analytic. In the case of the latter we will use notation due, in part to Oprea which we will define here for convenience. Letting $U = \bn$ denote the unit normal and $u, v$ refer to the first and second parameters of a simple surface, we have
\begin{align*}
  E &= g_{11}\\
  F &= g_{12} = g_{21}\\
  G &= g_{22}\\
  l &= \bx_{uu}\cdot U = L_{11}\\
  m &= \bx_{uv}\cdot U = L_{12} = L_{21}\\
  n &= \bx_{vv}\cdot U = L_{22}\\
\end{align*}

% Connection to complex analysis through harmonic equations

