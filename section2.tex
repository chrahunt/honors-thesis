%!TEX root=thesis.tex
\section{Minimal Surfaces}

% TODO: Citation
In the language of the previous section our examination of minimal surfaces would be something of a slog. The properties and intuition that we will build up will cover a wide range of approaches from the strictly geometric (in which case, for ease of understanding we will use the more intuitive notation of the former) to analytic. In the case of the latter we will use notation due, in part to Oprea which we will define here for convenience

% E, F, G, l, m, n, shape operator, H, K, etc.
\subsection{Minimal Surfaces in the Language of Surface Theory}

\subsection{Properties of Minimal Surfaces}

\subsection{Minimal Surfaces in the Language of Complex Analysis}