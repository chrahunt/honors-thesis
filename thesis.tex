\documentclass[11pt]{article}
\usepackage{amsmath, amsthm, amsfonts, amssymb, latexsym}
\usepackage[utf8]{inputenc} % For handling special characters.
\usepackage{tensor} % For properly aligned mixed upper/lower indices.
\usepackage[mathscr]{euscript} % For script font.
\usepackage{graphicx} % For pictures.
\usepackage{tikz} % For graphics.
\usepackage{multicol} % For equation formatting.

\usetikzlibrary{tikzmark,arrows,patterns,shapes,backgrounds,positioning}

%  Theoremstyle declarations.

\newtheorem{thm}{Theorem}
\newtheorem{cor}[thm]{Corollary}
\newtheorem{lem}[thm]{Lemma}
\newtheorem{prop}[thm]{Proposition}
\newtheorem{claim}[thm]{Claim}
\newtheorem*{unno_claim}{Claim}

\theoremstyle{definition}
\newtheorem{defn}[thm]{Definition}
\newtheorem{ex}[thm]{Example}
\newtheorem*{quest}{Question}
\newtheorem*{unno_ass}{Assumption}
\newtheorem*{unno_notat}{Notation}

\theoremstyle{remark}
\newtheorem*{unno_rem}{Remark}
\newtheorem*{unno_note}{Note}

% For page width.  Do not disturb.
\oddsidemargin=0.0in \evensidemargin=0.0in \textwidth=6.5in

% For paper height.  Do not disturb.
\topmargin=-0.5in \textheight=9.0in

% For no paragraph indents. 
% \setlength{\parindent}{0pt}

% Various symbol commands.
\newcommand{\scr}[1]{\mathscr{#1}}
\newcommand{\sU}{\scr{U}}
\newcommand{\sV}{\scr{V}}
\newcommand{\sA}{\scr{A}}
\newcommand{\bX}{\mathbf{X}}
\newcommand{\bx}{\mathbf{x}}
\newcommand{\bY}{\mathbf{Y}}
\newcommand{\by}{\mathbf{y}}
\newcommand{\bT}{\mathbf{T}}
\newcommand{\bS}{\mathbf{S}}
\newcommand{\bB}{\mathbf{B}}
\newcommand{\bn}{\mathbf{n}}
\newcommand{\bN}{\mathbf{N}}
\newcommand{\bg}{\boldsymbol{\gamma}}
\newcommand{\ba}{\boldsymbol{\alpha}}
\newcommand{\bk}{\boldsymbol{\kappa}}
\newcommand{\bt}{\boldsymbol{\tau}}
\newcommand{\zero}{\mathbf{0}}

\newcommand{\RR}{\mathbb{R}}
\newcommand{\CC}{\mathbb{C}}

% TODO: Weingarten Map L
\newcommand{\wg}{L}
\newcommand{\ptl}{\partial}
\newcommand{\eps}{\varepsilon}
\newcommand{\del}{\partial}

\newcommand{\Ck}{C^k}
\newcommand{\inv}{^{-1}}
\newcommand{\ra}{\rangle}
\newcommand{\la}{\langle}
\newcommand{\norm}[1]{\lVert#1\rVert}

% Tensors.
\newcommand{\Cfl}{\tensor{\Gamma}{_{ij}^k}}
\newcommand{\cfl}[3]{\tensor{\Gamma}{_{#1#2}^#3}}
\newcommand{\Rmn}{\tensor{R}{_{i}^l_{jk}}}
\newcommand{\rmn}[4]{\tensor{\Gamma}{_{#1}^#2_{#3#4}}}
\newcommand{\Llk}{\tensor{L}{^l_k}}
\newcommand{\llk}[2]{\tensor{L}{^#1_#2}}
\newcommand{\Krn}{\tensor{\delta}{_i^j}}
\newcommand{\krn}[2]{\tensor{\delta}{_#1^#2}}

\renewcommand{\Re}{\text{Re}\,}
\renewcommand{\Im}{\text{Im}\,}

% Other commands.
\newcommand{\buildset}[2]{\left\{#1\;\middle\vert\;#2\right\}}

%%%%%%%%%%%%%%
%% Graphics %%
%%%%%%%%%%%%%%

% Section 1 graphic.
% Node background pattern declarations.
\pgfdeclarepatternformonly{right crosshatch}{\pgfqpoint{-1pt}{-1pt}}{\pgfqpoint{4pt}{4pt}}{\pgfqpoint{3pt}{3pt}}%
{
  \pgfsetstrokeopacity{0.3}
  \pgfsetlinewidth{0.2pt}
  \pgfpathmoveto{\pgfqpoint{3.1pt}{0pt}}
  \pgfpathlineto{\pgfqpoint{0pt}{3.1pt}}
  \pgfusepath{stroke}
}

\pgfdeclarepatternformonly{left crosshatch}{\pgfqpoint{-1pt}{-1pt}}{\pgfqpoint{4pt}{4pt}}{\pgfqpoint{3pt}{3pt}}%
{
  \pgfsetstrokeopacity{0.3}
  \pgfsetlinewidth{0.2pt}
  \pgfpathmoveto{\pgfqpoint{0pt}{0pt}}
  \pgfpathlineto{\pgfqpoint{3.1pt}{3.1pt}}
  \pgfusepath{stroke}
}

% Section 3 graphic.
\renewcommand{\tikzmark}[2]{\tikz[overlay,remember picture,baseline=(#1.base)] \node (#1) {#2};}

\graphicspath{ {/figures/} }

\begin{document}

%%%%%%%%%%%%%%%%
%% Title Page %%
%%%%%%%%%%%%%%%%

% Do not disturb the spacing parameters.

\protect{\thispagestyle{empty}}

\begin{center}

\LARGE

\phantom{1}

\vspace{1.5in}

\sc Surface Theory, Miminal Surfaces, and Weierstrass Representations \normalfont

\Large

\vspace{1.0in}

Christopher A.\ Hunt

\vfill

submitted in partial fulfillment of the requirements for Honors in Mathematics at the
University of Mary Washington

\vspace{0.5in}

Fredericksburg, Virginia

\vspace{0.5in}

April 2014

\normalsize

\vspace{1.0in}

\end{center}

\pagebreak

%%%%%%%%%%%%%%%%%%%%
%% Signature Page %%
%%%%%%%%%%%%%%%%%%%%

% Do not disturb the spacing parameters.

\protect{\thispagestyle{empty}}

\phantom{1}

\vspace{0.5in}

\noindent  This thesis by \textbf{Christopher A.\ Hunt} is accepted in its present form as satisfying the thesis requirement
for Honors in Mathematics.

\vspace{1.0in}

% Thesis advisor FIRST, then readers alphabetically.

\begin{tabular}{p{1.0in}l}
\sc Date & \sc Approved \\[.65in]
\rule{0.85in}{0.5pt} & \rule{2.6in}{0.5pt}\\
& Yuan-Jen\ Chiang, Ph.D.\\    % Advisor
& thesis advisor\\[.65in]
\rule{0.85in}{0.5pt} & \rule{2.6in}{0.5pt}\\
& Randall D.\ Helmstutler, Ph.D.\\  % Reader #1
& committee member\\[.65in]
\rule{0.85in}{0.5pt} & \rule{2.6in}{0.5pt}\\
& Jangwoon\ Lee, Ph.D.\\   % Reader #2
& committee member
\end{tabular}

\pagebreak

%%%%%%%%%%%%%%
%% Contents %%
%%%%%%%%%%%%%%

% This creates the table of contents, but does not paginate it.  Do not disturb.

\phantom{1}

\vspace{0.5in}

\tableofcontents{\protect\thispagestyle{empty}}

\pagebreak

\setcounter{page}{1}

%%%%%%%%%%%%%%
%% Abstract %%
%%%%%%%%%%%%%%

\begin{abstract}
% TODO: Write abstract
  \noindent  We construct surfaces, techniques for analyzing these surfaces, then we analyze surfaces using these techniques. Along the way we note some interesting properties including a few which may be surprising (unexpected intrinsic invariants!). Minimal surfaces get our consideration after, and while on the surface they appear as innocuous creatures satisfying a particular property (zero mean curvature), the discussion that follows will show otherwise. Minimal surfaces lie at the intersection of a number of different disciplines, and here we will touch but a few. The Weierstrass Representations, where we take our leave, will have us connecting Minimal Surfaces and Complex Analysis.
\end{abstract}

\vspace{0.25in}

%%%%%%%%%%%%%%%%%%%%%%
%% Exposition Below %%
%%%%%%%%%%%%%%%%%%%%%%

{%!TEX root=thesis.tex
\section{Local Surface Theory}

Investigating the properties of surfaces will form the basis on which our journey towards minimal surfaces rests. We will go through the perils inherent in defining surfaces, and then how we use calculus on these surfaces to approximate the behavior at a point and extract information about the surface by extending on that approximation. Along the way we will make and derive various tools used for analyzing these surfaces, some of which we will bring along with us to Minimal Surfaces, others we will not, but they are interesting nonetheless and help illustrate some of the connections between Differential Geometry and various other branches of mathematics. We will then look at parallels between these surfaces in $\RR^3$ and higher dimensional manifolds, though only briefly.

\subsection{Defining Surfaces in $\RR^3$}

Defining surfaces is not particularly difficult, but our methods require a different set of considerations. We will first define a simple surface, or coordinate patch, which represents a small open section of a larger surface. After considering that, we build surfaces using an open covering of these coordinate patches, with a special requirements on their intersections which, we will see, is necessary to ensure our tools later on function properly.

}
\pagebreak
{%!TEX root=thesis.tex
\section{Minimal Surfaces}

% TODO: Citation
In the language of the previous section our examination of minimal surfaces would be something of a slog. The properties and intuition that we will build up will cover a wide range of approaches from the strictly geometric (in which case, for ease of understanding we will use the more intuitive notation of the former) to analytic. In the case of the latter we will use notation due, in part to Oprea which we will define here for convenience

% E, F, G, l, m, n, shape operator, H, K, etc.
\subsection{Minimal Surfaces in the Language of Surface Theory}

\subsection{Properties of Minimal Surfaces}

\subsection{Minimal Surfaces in the Language of Complex Analysis}}
\pagebreak
{%!TEX root=thesis.tex
\section{Weierstrass Representation for a Minimal Surface}

\subsection{Background}

\subsection{Weierstrass Representation}

\subsection{Examples}}
\pagebreak

%%%%%%%%%%%%%%
%% Bib Data %%
%%%%%%%%%%%%%%

\pagebreak  % This pagebreak is to occur RIGHT BEFORE the bib call-up.  Do not disturb.

\addcontentsline{toc}{section}{References}

% De-comment the next two lines when ready to compile the bibliography.  Use amsplain style.

\bibliographystyle{alpha}
\bibliography{thesis}
%\bibliographystyle{amsplain}
%\bibliography{my bib database filename without the .bib extension}

% De-comment the next line to call up the entire database, whether cited or not.
\nocite*

\end{document}
